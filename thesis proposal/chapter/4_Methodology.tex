\section*{Methodology}


Extract the key point of your research and formulate questions to answer them. Put the research questions in bullet form (RQ1, RQ2, …), and for each one, justify why it is important to answer it. If you are conducting engineering research instead of empirical research, it is better to formulate research objectives, research questions can then be used in the evaluation part.\\

\noindent If possible, state hypotheses and describe what led to these hypotheses. You should be able to interpret all possible outcomes with respect to your hypotheses and/or questions, no matter the fact, if they are concurrent with your assumption or not.\\

\noindent Describe your contribution with respect to concepts, theory, and technical goals. Ensure that the scientific and engineering challenges stand out so that the reader can easily recognize that you are planning to solve an advanced problem.\\

\noindent Then describe the methods and techniques used to solve or form the basis of the approach. Various scientific approaches are appropriate for different challenges and project goals. Outline and justify the ones that you have selected. For example, different research methods are, design science \cite{f99b6b96-d02d-3b52-a8ae-813e94bb93eb}, case study \cite{guideline}, survey \cite{grover1997tutorial} and experiment \cite{basili1986experimentation}. These are just a few, there are many more. Every different kind of problem requires a different approach/method.\\

\noindent Write down the main steps of your methodology in bullet format. Think about these steps as those you will perform over the whole duration of your thesis project. Sometimes, especially in engineering research, you perform steps in iterations; if so, then indicate that.
