\section*{Introduction}
% The introduction serves a concise overview, aiming to highlight the project’s significance.
% A good introduction will answer these questions and key points.\\
% \textbf{Significance}, it should be clearly stated why it is crucial to address the challenge, what new knowledge can be gained from it? How can the new knowledge be applied, mention a example.\\
% \textbf{Related work}, how is your research related to previous research. Describe the context of the challenge, this establishes the foundation of the problem.\\
% \textbf{Clear problem statement}, clearly define the problem, why is it important, what is the foundation?\\
% \textbf{Methodology}, briefly describe your methodology, without going into too much detail. The reader should understand the approach how you address the problem.\\
% \textbf{Novelty and Goals}, describe what is your goal, what is the novelty of your approach or solution?\\
% In summary, the introduction should briefly describe the whole thesis, serving as a roadmap for the whole project. The following sections describe the most important points in more detail.

The introduction provides a concise overview of the thesis, also aiming to highlight its significance. A good introduction will cover the following points.
\begin{itemize}
	\item \textbf{Context.} Describe the context of your thesis project in one or two paragraphs. This establishes the foundation of the problem addressed. The context describes the larger area your project concerns, and it embeds the following problem description, so that one knows in what area and under what circumstances the problem exists. Already use references here to highlight important general literature. Describe the context in a way that motivates working on this area. 
	\item \textbf{Problem.} Clearly define the problem, why it is important, and what is its foundation. It should be clearly stated why it is crucial to address the challenge and what new knowledge can be gained from it. Also, how can the new knowledge be applied? Mention an example.
	\item \textbf{Goals and novelty.} Now describe what the goal of your thesis project is. Especially explain its novelty, perhaps also the novelty of your approach or solution. It is important that the goal is significant and that it is not about the research method itself. For instance, a goal is never a study, but to determine some specific knowledge or to design or realize a solution to a research problem. 
	\item \textbf{Methodology.} Briefly describe your methodology, without going into too much detail. The reader should understand the approach of how you address the problem. What results do you expect, and what is your contribution? It should be clearly stated who is expected to benefit from the results of the thesis project.
	\item \textbf{Expected results and contribution.} Very briefly, describe the expected results and concrete contributions. The latter can be, for instance, a tool, a dataset, empirical data, or a new method. Very briefly, also define and delimit the specific research area and explain how your research relates to it. How is your research related to previous research?
\end{itemize}

\noindent In summary, the introduction should briefly describe the whole thesis, serving as a roadmap for the whole project. The following sections describe the most important points in more detail.



%\textbf{Context}, describe the context of the challenge, this establishes the foundation of the problem. Describe the motivation of your thesis project. Define and delimit the specific research area and explain how your research is applicable. How is your research related to previous research.\\
%\textbf{Problem}, clearly define the problem, why is it important, what is the foundation? It should be clearly stated why it is crucial to address the challenge, what new knowledge can be gained from it? How can the new knowledge be applied, mention an example.\\
%\textbf{Goals and novelty}, describe what is your goal, what is the novelty of your approach or solution?\\
%\textbf{Methodology}, briefly describe your methodology, without going into too much detail. The reader should understand the approach of how you address the problem. What results do you expect, and what is your contribution? It should be clearly stated who is expected to benefit from the results of the thesis project.\\
%In summary, the introduction should briefly describe the whole thesis, serving as a roadmap for the whole project. The following sections describe the most important points in more detail.

